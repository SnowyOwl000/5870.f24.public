\item\problemnumber{GT C}{1}{14}{-}{.}
A $n$-degree polynomial $p(x)$ is an equation of the form
$$p(x)=\sum_{i=0}^na_ix^i$$
where $x$ is a real number and each $a_i$ is a constant.
\begin{list}{\textbf{\alph{enumii}}}{\usecounter{enumii}}
\item Describe a simple $O(n^2)$-time method for computing $p(x)$ for a
particular value of $x$.
\item Consider now a rewriting of $p(x)$ as
$$p(x)=a_0+x\left(a_1+x\left(a_2+x\left(a_3+\cdots+x\left(a_{n-1}+xa_n\right)
\cdots\right)\right)\right)$$
which is known as \emph{Horner's method}. Using the $O(\,)$ notation,
characterize the number of multiplications and additions this method of
evaluation uses.\\[12pt]
\end{list}
\vskip12pt
\ifanswers
\textcolor{blue}{
\textbf{Answer:}\\
\begin{list}{\textbf{\alph{enumii}}}{\usecounter{enumii}}
\item Part a answer goes here
\item Part b answer goes here
\end{list}
}
\newpage
\fi
