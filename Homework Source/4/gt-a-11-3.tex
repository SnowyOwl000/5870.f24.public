\item\problemnumber{GT A}{11}{3}{-}{.}
In some multi-objective optimization problems (such as that exemplified by the
choosing of hotels based on the sizes of their pools and quality scores of their
restaurants), we may have different kinds of constraints involving the variables
instead of the desire to avoid points dominated by others. Suppose, for example,
that we have a two-variable optimization problem where we have a set, $C$, of
constraints of the form $$y\ge m_ix+b_i$$ for real numbers $m_i$ and $b_i$, for
$i=1,2,\ldots,n$. In such a case, we would like to restrict our attention to
points that satisfy all the linear inequality constraints in $C$. One way to
address this desire is to construct the \emph{upper envelope} for $C$, which is
a representation of the function $f(x)$, defined as $$f(x)=\max_{1\le i\le n}
\lbrace m_ix+b_i\rbrace$$ where the $(m_i,b_i)$ pairs are from the inequalities
in $C$. Equivalently, the upper envelope is a representation of the part of the
plane determined by the intersection of all the halfplanes determined by the
inequalities in $C$. If we consider how this function behaves as $x$ goes from
$-\infty$ to $+\infty$, we note that each inequality in $C$ can appear at most
once during this process. Thus, we can represent $f$ in terms of a set $S=
\lbrace(a_1,b_i,i_1),(a_2,b_2,i_2),\ldots,(a_k,b_k,i_k)\rbrace$, such that each
triple $(a_j,b_j,i_j)\in S$ represents the fact that interval $[a_j,b_j]$ is a
maximal interval such that $f(x)=m_{i_j}x+b_{i_j}$. Describe a $\Theta(n\log
n)$-time algorithm for computing such a representation $S$ of the upper envelope
of the linear inequalities in $C$.\\[12pt]
\ifanswers
\textcolor{blue}{
\textbf{Answer:}\\[6pt]
Answer goes here
}
\newpage
\fi
